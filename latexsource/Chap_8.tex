\chapter{The Cryosphere}\label{Cryosphere}
\section{Epigraph}\label{Cryosphere_Epigraph}
\epigraph{We were now revelling (sic) in the indescribable freshness of the Antarctic that seems to permeate one's being, and which must be responsible for that longing to go again which assails each returned explorer from polar regions. Sir Ernest Henry Shackleton, \textit{The Heart of the Antarctic}, 1914, pg. 44} 

\section{Core Concepts} \label{Cryosphere_Core Concepts}
\begin{itemize}
	\item	The Cryosphere---the part of Earth dominated by ice in all its forms---is the smallest, most volatile and most endangered of all Earth's spheres.
	\item	Because ice and snow have relatively high albedos, the cryosphere exerts exceptional power on global and regional energy balance over many time scales---from months to millions of years.
	\item Earth history has been punctuated by multiple episodes of profound glaciation, including the current period of Ice Ages, which began roughly 33 Myr.
	\item	Current episodes of Ice Ages begin and end periodically, due to periodic variations in earth's orbit which change the insolation incident on the northern hemisphere. These Milankovich cycles are due to the gravitational effects of other planets on Earth's orbit.    
	\item	Global climate change is amplified in the poles, and profound changes to the Cryosphere are certain within your lifetime
\end{itemize}

\section{Opening Problem} \label{Cryosphere_Opening Problem}
What's the highest point in \emph{your} country? How tall is it? And for how long? We tend to think of the world's high points as permanently lofty, fixed edifices towering above the rest of the country. Sweden's high point\footnote{The original article (in Swedish) is \href{http://www.mynewsdesk.com/se/su/pressreleases/kebnekaises-sydtopp-blir-laegre-aen-nordtoppen-i-augusti-2611533}{here}} is the southern peak of Kebnekaises (``Cauldron Crest'' in the local Sami language); the peak is not rock but a glacier (shown in Figure \ref{fig:kebenkaises}. Professor Gunhild Ninis Rosqvist has been measuring the height of Kebnekaises for years as part of her ongoing research on climate change. The summer of 2018 CE brought record heat to Europe, including the Arctic reaches of northern Sweden and Kebnekaises. During just July of 2018, Dr. Rosqvist's measurements showed that Kebnekaises melted from \SI{2101}{\metre} to \SI{2097}{\metre} above sea level, a loss of by \SI{4}{\metre} at the astonishing rate of \SI{14}{\centi\metre\per\day}. Dr. Rosqvist noted that ``The result of the hot summer will be a record loss of snow and ice in the mountains'' So much of the glacier melted that the northern peak of Kebenkaise became Sweden's high point, at least until the next winter, when new snow fall might bring the southern peak back to its former prominence. Seasonal melting of snow and ice is part of the natural cycle of the cryosphere---the part of Earth made of ice. But the unprecedented melting experienced by Kebenkaise in the summer of 2018 is unfortunately just one example of the large-scale melt-back of Earth's cryosphere. This melting is not just a local inconvenience, but a regional and global one, affecting everything from the stability of individual homes, to water supply for major cities, to the planet's equilibrium temperature. As Professor Shawn Marshall noted\footnote{Marshall, S. J. (2011). The cryosphere. Princeton University Press, pg. 236} ``The atlas of the world will be redrawn as coastal boundaries, navigable waterways, river systems, and population centers shift in the coming decades.'' The problem we face is simple: how do we mitigate or adapt our society to these real-time changes? 



\section{Introduction: the importance of Cryosphere---climate feed backs} \label{Feedbacks}
Uniquely among, well, the entire known Universe, Earth's average surface temperature (\SI{288}{\kelvin}, \SI{15}{\celsius}, $60^{\circ}F$) and pressure are near the freezing point of water (\SI{273}{\kelvin}, \SI{0}{\celsius}, $32^{\circ}F$). At various times, in various places and for various durations, temperatures fall below that freezing point and in a complex and still incompletely understood process\footnote{Pradzynski, C. C., Forck, R. M., Zeuch, T., Slav'icek, P., \& Buck, U. (2012). A Fully Size-Resolved Perspective on the Crystallization of Water Clusters. Science, 337(6101), 1529. \href{https://doi.org/10.1126/science.1225468}{Link}} liquid water freezes into solid ice. The sum total of all that ice at any time is Earth's \emph{Cryosphere}. If you look back at the previous few chapters, you'll find a theme: our examination of each sphere starts with a monologue about the sphere's role in moving heat from the Equator to the Poles, moderating Earth's surface temperatures. Not this chapter. The Cryosphere is too small, too un-massive, too static and too ephemeral to move any substantial heat around the planet. On local and regional scales, the Cryosphere acts like a thermal battery, storing heat for later, slower release, and moderating local and regional temperature changes. But the climatically important role of the Cryosphere lies at the core of its ephemerality: newly-fallen snow and ice are bright and reflective, so they have unusual high albedo. Snow and ice can reduce Earth's equilibrium surface temperature, and hence make the additional snow and ice more likely. These feedbacks reach into the heart of the carbon cycle, so the cryosphere-albedo-surface temperature feedback is one of the strongest and most important in climatology. For the past 2.6 million years these feedbacks have repeatedly plunged Earth into Ice Ages\footnote{Maslin, M. A., \& Brierley, C. M. (2015). The role of orbital forcing in the Early Middle Pleistocene Transition. Quaternary International, 389, 47-55. \href{https://doi.org/10.1016/j.quaint.2015.01.047}{Link}}, and even affected the timing and path of human evolution and migration.\\

The feedback begins with increases in snow and ice cover \ref{fig:cryo_albedo_feedback}, which reduces Earth's average albedo. It doesn't take much snow to raise Earth's albedo significantly: coating just 1\% of Earth's surface with snow (about the amount shown in Figure \ref{fig:jonas}) would raise average albedo by nearly 2 percentage points\footnote{Roughly, for an Earth ($A_\Earth=0.30$) fraction $f$ newly covered with snow ($A_{snow}=0.80$), $A_{new} \approx (1-f)A_\Earth+fA_{snow}$ so $A_{new} \approx 0.305$ and $dA=0.005=5/3\%$.}. This albedo change leads to a change in Earth's equilibrium temperature $T_{eq}$---assuming the snow never melted or darkened, reasonable in this thought experiment--- of \SI{0.5}{K} (see \footnote{For $T_{eq}=\left(\frac{S(1-A)}{4\sigma}\right)^{1/4}$, $\left(dT_{eq}\right)_S=-\frac{1}{4}\frac{1}{1-A}T_{eq}dA=\SI{0.46}{\kelvin}$}), which lowers the surface temperature, $T_{surf}$, which increases the Cryosphere's stability formation of more snow and ice. Over time, average temperatures in the oceans fall, which draws $CO_2$ from the atmosphere and into the oceans, reducing the greenhouse effect, further lowering $T_{surf}$ and altering the carbon cycle! A little bit of snow can, over a few centuries, make a difference of the world. The intricate relationships between the Cryosphere and other aspects of the Earth System are shown in Figure \ref{fig:cryo_concept_map}. We'll explore the Cryosphere in reverse, briefly noting the small and temporary components, and then dwelling on glaciers and ice sheets. These will lead us naturally into our examination of the Ice Ages and the factors---both here on Earth and in the greater Solar System---controlling them. 

 

 
\section{The Seasonal Cryosphere and Arctic Amplification}  \label{Arctic Amplification}
On January 26, 2016, about a third of United States' population became citizens of the Cryosphere\footnote{Putting the January 22-24 Snowstorm in Historical Context, February 18, 2016, \href{https://www.ncdc.noaa.gov/news/january-22-24-2016-snowstorm-in-historical-context}{link}, accessed 14 August 2018}. Over the previous 2 days, as much as \SI{81}{\centi\metre} (nearly\footnote{Winter Storm Jonas: Where Does it Rank Historically? January 24 2016 12:15 PM EDT, \href{https://weather.com/storms/winter/news/winter-storm-jonas-rank-in-history}{link} accessed 14 August 2018} 32 inches!) of snow fell in a wide swath (see Figure \ref{fig:jonas} from Arkansas to Massachusetts. Particularly outside the polar regions, the Cryosphere is only marginally stable has temperatures rise and fall on daily and seasonal cycles. Unlike all the other spheres, the Cryosphere variable over time and space. Superimposed on these cycles is the long term global warming, which is driving rapid, real-time erosion of much of the Cryosphere. Recall these variations as we tour the existing Cryosphere, keeping in mind that all these values are for this moment in Earth's long history. Table \ref{tab:wti} and Figure \ref{fig:wheres_the_cryo} illustrate our discussion.
\subsection{Snow and ice on land}
\subsubsection{Snow}\label{snow_talk} Occasionally the Cryosphere reaches in to the atmosphere, as snow and its brethren. As beautiful and tranquil as a snowfall can be (Figure \ref{fig:snowfall}), the processes leading to snow involve titanic forces in the atmosphere.  Because the amount of water dissolved in air decreases exponentially with temperature, (Figure (reference to chapter 5)), most snow forms when relatively warm moist air is cooled to temperatures below freezing. (This is why snow is actually quite rare in perennially cold places, such as Antarctica!) the most common places to get this combination is in orographic uplift, and large, mid-latitude storm systems. Air cools as it rises, so air masses running up against mountains will cool as they lift over the mountains, dropping their water on the upwind side of the range. As you can see in Figure \ref{fig:orographic} the snow falls almost completely on the ``uphill'' side of the range. The other common source of regional snow are winter storms in mid-latitude temperate areas. These storms produce snow when cold, polar air masses meet warmer, wetter air masses in low-pressure areas. The warm moist air circulates (in the Northern Hemisphere) counter-clockwise over the cold air, cooling and producing snow. This was the cause of the massive snow shown in Figure \ref{fig:jonas}. During the height of the Northern Hemisphere winter, snow can cover up to 30\% of the the continents, as shown by the gray areas in Figure \ref{fig:seasonalsnow}.\\

The relationship between continued global warming and snow fall is complicated, both physically\footnote{Kunkel, K. E., Robinson, D. A., Champion, S., Yin, X., Estilow, T., \& Frankson, R. M. (2016). Trends and extremes in northern hemisphere snow characteristics. Current Climate Change Reports, 2(2), 65-73.} and politically. In low-latitude and low-lying areas, the limits of stable snowfall will move to higher latitudes and elevations. SO we expect snow covered areas to decrease over time. But, warmer air contains more water than colder air, about 6\% more for each \SI{+1}{\kelvin} of warming. For places that are cold but currently dry---such as the Great Plains of the United States and Canada---for the short term global warming may bring \emph{increased} snow. Even in aworld where climate is changing rapidly, weather still happens, and even places where snow is rare will still enjoy a making snowballs. This counter-intuitive result has repeatedly been at the center of political (snow) storms\footnote{For example, Bump, P. (2015). Jim Inhofe's snowball has disproven climate change once and for all. Washington Post, February 26, 2015. \href{https://www.washingtonpost.com/news/the-fix/wp/2015/02/26/jim-inhofes-snowball-has-disproven-climate-change-once-and-for-all/?utm_term=.d10d986358c5}{Link}}, but again reminds us of the difference between weather and its long term average, climate.\\
From a climatic perspective, snow \emph{is} getting less common. For example, since the early 1920s CE the average snow cover in the Northern Hemisphere has decreased\footnote{AR5, Chapter 4, page 320, ``Over the longer period, 1922-2012, data are available only for March and April, but these show a 7\% [very likely, 4.5\% to 9.5\%] decline''} by at least 7\%, due to the the gradual increase in surface temperatures (~\SI{+1.1}{\kelvin}, or $2^{\circ}F$). Figure \ref{fig:sce} illustrates both sides of the change in snow. Areas in brown on the map recorded less snow in  of the Another statistic makes just how fast snow cover is changing, particularly because of earlier springtime melting. The Northern Hemisphere snow season---the length of time snow lays on the ground---has declined by \emph{20 days} since the winter of 1972-73 CE\footnote{Choi et al., 2010, as quoted in AR5 op cit, page 358.}.  



\subsubsection{Rivers and lakes} Together, rivers and lakes make up only 4\% of Earth's surface\footnote{Rivers: Allen, G. H., \& Pavelsky, T. M. (2018). Global extent of rivers and streams. Science, eaat0636. \href{https://doi.org/10.1126/science.aat0636}{Link}. Lakes: Cael, B. B., Heathcote, A. J., \& Seekell, D. A. (2017). The volume and mean depth of Earth's lakes, Geophysical Research Letters, 44(1), 209-218. \href{https://doi.org/10.1002/2016GL071378}{Link}}, and so you might reasonably wonder why they'd be an important part of the Crysophere. But about $2/3$ of those rivers and lakes freeze up at least partially during the winter\footnote{Ionita, M., Badaluta, C.-A., Scholz, P., \& Chelcea, S. (2018). Vanishing river ice cover in the lower part of the Danube basin - signs of a changing climate. Scientific Reports, 8(1), 7948. href{https://doi.org/10.1038/s41598-018-26357-w}}, and the history of this freezing is a rich source of data on global climate change. With medium confidence, the IPCC concluded\footnote{AR5 op cit. page 361} that the ice season in the Northern Hemisphere is getting shorter, at 30 to 40 days over the last century. As we saw with snow cover, the greatest reductions are during the Spring melt season, and reflect how the edge of the Cryosphere is rapidly eroding in our warming world. 

\subsection{Sea Ice}
At noon on August 15, 1990, your author learned first-hand---well, first foot---about sea ice. He and two much smarter colleagues had set off that morning in a small inflatable raft to return to our base camp and the helicopter that would fly us from our study area in Northern Svalbard to the fascinating town of Langyerban. Our destination was just across the Wijdefjorden, and the day before had taken a few minutes of sun-filled fun. Overnight, winds had blown in sea ice, loose sheets of frozen sea water a few inches thick. In between the floes of ice were tiny little lanes of clear water, and we had been patiently and unsuccessfuly following one lead (pronounced \textit{leed}) after the other to try and cover the last mile between us and our ride home. After an hour of mounting frustration we came to yet another dead end: the sea ice had closed up and we were surrounded by ice. Perhaps 10 or 20 feet away, another lead opened up into a clear patch of sea and an easy ride home. So I laid on the edge of the raft and tried pushing the thin floes out of our way. It worked, for a moment, until I lost hold of the raft and fell into the ice-cold water. Fortunately, I was wearing a full ``immersion suit,'' a bright-orange, rather cumbersome piece of gear that insulates the wearer from cold water. Bobbing in the water, we all laughed at my stupidity of falling overboard, and then I swam through a tiny crack in the ice, arms flailing windmill fashion, opening up a path for the boat. After dragging us to the other side of the floe, we made it to base in plenty of time for our ride home. \\

Sunlight fades pretty quickly as summer turns to autumn in Earth's polar regions. At the North Pole, the sun sets---for six months---on September 21. But over all the Arctic Ocean the nights become longer and colder as energy received from the Sun plummets\footnote{NSIDC \href{http://nsidc.org/arcticseaicenews/2017/09/arctic-sea-ice-at-minimum-extent-2/}{Link}} starting in mid-September. The air quickly cools below the sea surface temperature, and the ocean rapidly looses heat to the overlying atmosphere. Once the surface cools to about \SI{-1.5}{\celsius} ($30^{\circ}$F) ocean water begins to freeze in to \emph{sea ice} (see Figure \ref{fig:seaice_anncycle}, upper panel), a mixture of water ice sprinkled with pockets of salty water, air bubbles, and even particles of salt. This newly-formed sea ice slowly thickens from the bottom, reaching \SIrange{1}{2}{\metre} thick over the long polar winter. The ice is blown and dragged around the Arctic Ocean, sometimes for years. Collisions between older ice floes thickens the ice, locally forming ice ridges \SIrange{10}{15}{\metre} thick. The resulting sea ice covers portions of the the Arctic Ocean year-round, providing habitats for seals, polar bears, and the occasional submarine (Figure \ref{fig:sub_polar}). Once spring arrives, increasing duration and intensity of sunlight melts some of the ice, and the cycle starts anew . Although the seasons are reversed, the same annual cycle of sea ice growth and decay also happens in the Southern Hemisphere (see Figure \ref{fig:seaice_anncycle}, lower panel), with an extensive belt of sea ice growing around Antarctica.\\ 
     
The climatic effects of sea ice are profound, because it is a highly reflective but insulting lid. Sea ice has a much higher albedo than water, which keeps the poles cold (Figure \ref{fig:cryo_albedo_feedback}). But sea ice also insulates the relatively warm ocean from the frigid polar atmosphere, drastically slowing heat flow from the ocean to the atmosphere. This is an excellent example of how the Cryosphere slows heat flow within in the climate system. But because sea ice (and snow) are so thin, and their stability is so sensitive to temperature, they are unusually susceptible to tiny changes in heat flow and surface temperatures.\\

\subsubsection{Arctic Amplification}\label{aa} The entire Cryosphere, and particularly the seasonal parts in the Northern Hemisphere, are in the midst of rapid and worrisome changes. The Arctic is warming two to four times faster than the rest of the planet\footnote{Coumou, D., Di Capua, G., Vavrus, S., Wang, L., \& Wang, S. (2018). The influence of Arctic amplification on mid-latitude summer circulation. Nature Communications, 9(1), 2959. \href{https://doi.org/10.1038/s41467-018-05256-8}{Link}.}, a pattern called \emph{Arctic Amplification}. One specific example of this is shown in Figure \ref{fig:aa}, which demonstrates that surface temperature changes at the North Pole are about 3 times those of the entire globe. Arctic Amplification is largely the consequence of feedbacks shown in Figure \ref{fig:cryo_concept_map}. Increasing surface temperatures in the Arctic attack sea ice in two ways. Warmer temperatures means ice doesn't grow as far south as it used to, so the average area covered by ice is deceasing by about 1\% per year. Those higher air and ocean temperatures are also thinning the ice that's already there, reducing the most stable and thick part of the Arctic sea ice so strongly that its volume has decreased by 80\% in one human generation\footnote{Francis, J. A. (2017). Why Are Arctic Linkages to Extreme Weather Still up in the Air? Bulletin of the American Meteorological Society, 98(12), 2551-2557. \href{https://doi.org/10.1175/BAMS-D-17-0006.1}{Link}}! Figure \ref{fig:min_ice} (upper panel) shows the minimum annual extent of Arctic sea ice measured by satellite since 1979 CE. The minimum is clearly decreasing over time, even though year-to-year variations are significant. Most disturbing is that the loss is \emph{accelerating}, due again to the feedbacks driving Arctic Amplification. The lower panel of Figure \ref{fig:min_ice} is a reasonable model for the future of Arctic sea ice, with the solid and dashed red lines showing best-estimate and 90\% confidence interval for when the Arctic will be ice free in summer\footnote{The Arctic ocean will be ``effectively ice free'' at ~$1 Mkm^2$ of sea ice. This is a subtlety I believe worth skipping.}. As of August of 2018 CE, with the minimum still a month away, the model predicts an even chance that summer-time Arctic sea ice will vanish by 2040 CE, when most readers of this book will still be paying off their college loans. There's a 5\% chance of an ice-free Arctic Ocean by 2032 CE, just 14 years from the date of writing\footnote{These dates are based entirely on the observed monthly average sea ice extents, and so are robust to small variation in daily extent data. I use a second order polynomial in the fitting, which is preferred over a linear model (based on AIC) with an odds ration of ~32:1. The 2030-2040 range is in excellent agreement with Notz, D., \& Stroeve, J. (2016). Observed Arctic sea-ice loss directly follows anthropogenic CO2 emission. Science, 354(6313), 747-750.}. One hopes Santa Claus knows how to swim.\\
 
An ``unprecedented'' and ``scary'' example\footnote{Watts, Jonathan, 21 Aug 2018, Arctic's strongest sea ice breaks up for first time on record \href{https://www.theguardian.com/world/2018/aug/21/arctics-strongest-sea-ice-breaks-up-for-first-time-on-record}{Link}, accessed 24 August 2018} of the effects of Arctic Amplification on sea ice is shown in Figure \ref{fig:breakup}. The image is of the northern-most tip of Greenland, at \SIrange{80}{83}{\celsius} N latitude. This remote corner of the world once had the thickest, oldest sea ice in the Arctic; here winds and currents thrust ice against the coast, repeatedly stacking and re-stacking floes until the ice was as thick as \SI{20}{\metre} (70 feet). By carefully collecting and finding the age of driftwood along this coast\footnote{Funder, S., Goosse, H., Jepsen, H., Kaas, E., Kjaer, K. H., Korsgaard, N. J., ... Willerslev, E. (2011). A 10,000-Year Record of Arctic Ocean Sea-Ice Variability-View from the Beach. Science, 333(6043), 747-750. \href{https://doi.org/10.1126/science.1202760}{Link}}, European scientists have shown that this area of Greenland has had permanent, thick sea ice for at least the past 6000 years. No more. The area labeled ``Open Ocean'' in Figure \ref{fig:breakup} is where that thickest and oldest ice once sat. Thinned by the unusually hot winter and summer of 2018 CE, the ice was broken up by winds and scattered. The dark ocean waters will absorb an extra dose of energy, likely delaying the growth of new sea ice. According to Walt Meier\footnote{Watts, op cit.}, a senior research scientist at the US National Snow and Ice Data Center, warming, melting and thinning have reached ``even the coldest part of the Arctic with the thickest ice. So it's a pretty dramatic indication of the transformation of the Arctic sea ice and Arctic climate.''\\  

%quote below from Francis, J. A. (2017). Why are Arctic linkages to extreme weather still up in the air?. Bulletin of the American Meteorological Society, 98(12), 2551-2557.
%\begin{quotation}
	%It has been known for many years that the decline in sea ice extent accounts for only part of the rapid warming that has been observed in the Arctic atmosphere in recent decades, known as Arctic amplification (AA), and that the sea ice-induced warming is confined mostly to the lower atmosphere. Some estimates of the direct influence of reduced sea ice extent on AA are as low as 20\% (Perlwitz et al. 2015). Several other factors are known to contribute, such as thinning sea ice, to which Lang et al. (2017) ascribe about 37\% of observed surface warming, and declining spring snow cover, which augments high-latitude warming in late spring and summer (Estilow et al. 2015). Warming aloft is caused by a combination of heat trapped by additional water vapor from both local and remote sources (Porter et al. 2012), more abundant clouds, and latitudinally varying changes in atmospheric temperature profiles (Pithan and Mauritsen 2014; Burt et al. 2016). Any additional heat that finds its way to the Arctic is amplified by a variety of positive feedbacks in the region. Consequently, in modeling experiments that exclude natural or forced energy exchanges with more southerly zones, a large fraction of AA is missed. Other contributing factors may include model inaccuracies in boundary layer stratification and radiative transfer (Bintanja and Krikken 2016) that affect energy exchanges, as well as heat transported by ocean currents (Polyakov et al. 2017).
	%
	%Thus, while AA weakens the poleward gradient in the lower troposphere, tropical amplification (TA) strengthens the gradient in upper levels.
	%
	%The effects of climate change on extreme weather are a topic of intense scientific interest and of vital societal impact. Some of these effects are clear-such as more severe heat waves, more frequent heavy precipitation events, and more persistent droughts-but other less direct influences are still "up in the air." The role of a rapidly warming and melting Arctic is one of these factors that challenges present modeling capabilities and dynamical understanding. These limitations are now coming into focus as changes in the real world either confirm or oppose expectations based on simulations, offering avenues to resolve disputes in our understanding of Arctic-midlatitude linkages. Future work using targeted simulations based on suitable models, targeted experimental design, and relevant circulation metrics, as further described in Overland et al. (2016), will undoubtedly dissipate some of the cloudiness obscuring the impacts of a rapidly warming and melting Arctic on weather patterns in temperate latitudes.
%\end{quotation}

The loss of Arctic sea ice and snow might seem a distant problem, one relegated to the far northern corner of the globe and too distant to affect the world at large. Unfortunately, all that missing sea ice and snow means more heating of the Arctic Ocean and the surrounding land. This heat eventually warms the Arctic atmosphere, changing atmospheric circulation and increasing the severity of both winter storms and summer heat waves across the Northern Hemisphere. While this connection is still controversial\footnote{Francis, J. A. (2017). Why Are Arctic Linkages to Extreme Weather Still up in the Air? Bulletin of the American Meteorological Society, 98(12), 2551-2557. \href{https://doi.org/10.1175/BAMS-D-17-0006.1}{Link}}, the mechanism is increasingly well-understood\footnote{Coumou, D., Di Capua, G., Vavrus, S., Wang, L., \& Wang, S. (2018). The influence of Arctic amplification on mid-latitude summer circulation. Nature Communications, 9(1), 2959. \href{https://doi.org/10.1038/s41467-018-05256-8}{Link}; Mann, M. E., Rahmstorf, S., Kornhuber, K., Steinman, B. A., Miller, S. K., \& Coumou, D. (2017). Influence of anthropogenic climate change on planetary wave resonance and extreme weather events. Scientific Reports, 7, 45242.} During the winter, the warmer Arctic atmosphere weakens the prevailing westerlies in the temperate areas of the Northern Hemisphere. This allows frigid Polar air to sag southward, bringing cold temperatures and increased snowfall to Europe and North America. (This is an excellent example of how global warming can locally \emph{increase} snow fall, here by juxtaposing cold, dry polar air with warmer, moister temperate air.)\\
During the summer, the warmer Arctic atmosphere again alters the flow of the jet stream, allowing it to ``lock'' into a particularly wavy state. Areas south of the jet stream are bathed in relatively hot, humid air, which warms and dries the ground, creating devastating heat waves in those areas. Storms get weaker, and those storms tend gravitate toward the jet stream. This prevents rain from reaching the hot areas, plunging them into drought. The hot dry areas become even hotter and drier, which stabilizes the jet its loopy position. This feedback can lead to weeks of fatal heat waves in Europe and North America. The 2010 heatwave in Central Europe\footnote{Barriopedro, D., Fischer, E. M., Luterbacher, J., Trigo, R. M., \& Garc'ia-Herrera, R. (2011). The Hot Summer of 2010: Redrawing the Temperature Record Map of Europe. Science, 332(6026), 220. \href{https://doi.org/10.1126/science.1201224}{Link}}, killed 55,000 people in Russia alone, destroyed 25\% of its crops, and depressed the country's annual economic output by 1\%. As the inset of Figure \ref{fig:aa} shows, the Arctic is expected to warm by \SIrange{5}{10}{\celsius} ($9\ to \ 18^{\circ}F$) by the end of the century, likely leading to more frequent, hotter, wider-spread ``mega-heatwaves.''    


%Sea ice has crucial climate role. It annually covers/uncovers the oceans with a high-albedo blanket, profoundly changing the the flow of heat in climate system. But it is sensitive to that heat flow itself, so can change quickly as boundary conditions change.
%seasonal cycle
%bc of salinity, densest water is NOT \SI{4}{\celsius} but coldest water, ice forms from cold overlying air cooling surface
%sea ice freezes from ocean water from below; for that reason sea ice rarely grows thicker than 2 m...absent forces that mechanically thicken the ice. 
%But even multi-year ice is typically $< 5 m$ thick.
%first-year ice is half as salty as sea water; but older ice is far fresher, bc brine is rejected in freeze thaw cycles.
%Relative thinness of ice means it is susceptible to high frequency variations. 
\subsection{Permafrost}
Permafrost is the groundwater of the cryosphere---water-saturated soil which remains below freezing for at least a few years. In the Arctic, permafrost can be hundreds of meters (hundreds of yards) thick and older \footnote{Froese, D. G., Westgate, J. A., Reyes, A. V., Enkin, R. J., \& Preece, S. J. (2008). Ancient Permafrost and a Future, Warmer Arctic. Science, 321(5896), 1648-1648. https://doi.org/10.1126/science.1157525} than 700,000 years. Don't just think ground under your foot! Permafrost includes sediments under the shallow oceans of continental shelves. These areas were land exposed to the atmosphere during ice age maxima (see section \ref{ice_ages}, but have since been flooded as ocean s swelled from melting ice sheets. The amount of ice in permafrost is actually quite small (see Table \ref{tab:wti}, but frozen ground covers an astonishing quarter of all land area in the Northern Hemisphere (Figure \ref{fig:permafrost}). On the top, permafrost is stable only as long as air temperatures remain below freezing, \SI{0}{\celsius}. In some areas (increasingly many, given Arctic Amplification), summertime temperatures melt the upper 1 m or so of soil, but the underlying ice stays frozen and impermeable to water. The melt water collects at the surface, forming the boggy, but gorgeous and vast tundra, the characteristic collection of plants and animals of the Arctic. Permafrost is also melted \emph{from below}, by heat flow from Earth. Assaulted from both sides, permafrost is clearly in a precariously narrow zone of stability.\\ 
%http://old.grida.no/graphicslib/detail/the-cryosphere-world-map_e290 url for the bad map

\subsubsection{Permafrost Carbon Feedback} As we saw in Section \ref{aa}, temperatures in the Arctic are rising rapidly and snow cover is decreasing, both of which drive more heat into soil. Just like Northern Hemisphere snow and sea ice, permafrost is melting, and at an accelerating rate\footnote{AR5, Chap. 4.7.2}. Local and regional effects of this melting are severe: tundra disappears, buildings can collapse (Figure 2.7), and extensive erosion are all common occurrences. And while these changes are devastating to those suffering through them, the large-scale threat of melting permafrost is carbon stored in that melting soil. That carbon is the remains of organisms---largely plants---which died 10 kya. Frozen soil in the Arctic stores about one and half times the carbon in the atmosphere\footnote{Tarnocai, C., Canadell, J. G., Schuur, E. a. G., Kuhry, P., Mazhitova, G., \& Zimov, S. (2009). Soil organic carbon pools in the northern circumpolar permafrost region. Global Biogeochemical Cycles, 23(2). \href{https://doi.org/10.1029/2008GB003327}{Link}}, and this storage is threatened by permafrost melting. When cold and frozen the carbon isn't consumed by soil-dwelling organisms, but once thawed, the carbon is quickly utilized by organism and eventually expelled as carbon dioxide ($CO_2$) and methane ($CH_4$) into the atmosphere. These GHG increase temperatures and decrease snow cover, initiating a strong feedback that only accelerates the changes, as shown by the \textit{Permafrost} pathway in Figure \ref{fig:cryo_concept_map}. Reasonable estimates\footnote{Schaefer, K., Lantuit, H., Romanovsky, V. E., Schuur, E. A. G., \& Witt, R. (2014). The impact of the permafrost carbon feedback on global climate. Environmental Research Letters, 9(8), 085003. \href{https://doi.org/10.1088/1748-9326/9/8/085003}{Link}} suggest that by 2100 CE, the carbon released from thawing permafrost could increase average global warming by \SIrange{0.1}{0.5}{\kelvin}, a substantial amount on a global basis.\\
%But another source of potential C lies in the permafrost: organic material---the remains of dead plants and animals that died more than 10,000 years ago. Although this reservoirs has only half the C stored in clathrates ($9.5x10^2$ Gt C, from Mascarelli), it's \emph{all} in shallow, terrestrial PF. As the PF melts (which, as we've seen, is happening) this C is consumed by microbes and released as CH4 and/or CO2. The potential for this frozen C to affect climate is high: that frozen C is about the same as all of the C humans have emitted into the atmosphere!  
 
%loss is rapid now, as artic is warming quickly (AR5 says ``Permafrost temperatures have increased in most regions since the early 1980s (high confidence) although the rate of increase has varied regionally. The temperature increase for colder permanotfrost was generally greater than for warmer permafrost (high confinotdence). {4.7.2, Table 4.8, Figure 4.24}''
%includes both terrestrial and marine (continental shelf, exposed and cold at last glacial max, now flooded) ground! The latter is important source of methane hydrates (deserves a box); timescales for change are $10^1$ to $x10^4$ years; The shallow active surface layer (~1m, typically $<10$ m) has annual temperature variations \\

\subsubsection{A bit of good news: clathrates} By now, you might be reeling from the impacts of a warming cryosphere. So some good news: one potential source of GHGs frozen in the cryosphere is \emph{not} in imminent danger. Clathrates are ``cages'' of water molecules wrapped around gas molecules, generally methane ($CH_4$), and are found where cold water and methane are squeezed together by the weight of overlying sediments and water. Together the water and methane form an icy-looking deposit, which has so much methane (they concentrate the gas by a factor of 200) that it's flammable, as shown in Figure \ref{fig:clath_fire}. Clathrates are stable only in the deepest portions of permafrost, those insulated from temperature changes at the surface by hundreds of meters of soil. Warming will take centuries to propagate to this depth, and thus the prodigious quantities of methane stored here won't soon reach the surface\footnote{Ruppel, C. D., \& Kessler, J. D. (2017). The interaction of climate change and methane hydrates: Climate-Hydrates Interactions. Reviews of Geophysics, 55(1), 126-168. \href{https://doi.org/10.1002/2016RG000534}{Link}}.   

%\paragraph{Clathrates, permafrost, methane and global warming}
%Clathrates are ``cages'' of water molecules wrapped around gas molecules. Formation and stability require precise combination of temperatures and pressures: temperatures $< 25$ C and pressures $> 300-500 Mpa (=>)$ 300 to 500 m of water+seds). B/c molecules of gas are typically far apart, the clathrates concentrate the gas, by a factor of 180 times! As Ruppel points out (Ruppel, C. D. (2011) Methane Hydrates and Contemporary Climate Change. Nature Education Knowledge 3(10):29) a little melted clathrate is a lot of methane gas. Reasonable estimate for CH4 in clathrates is (Boswell and Collett, 2011) 15\% of C in shallow organic reservoirs, roughly $~1.8x10^3$ Gt C!  There is some evidence and much anxiety that increasing ocean and land temperatures and changing sea levels could destabilize this methane, leading to increased GHG concentrations. Release of $<0.2\%$ of this would double $[CH_4]_atm$! This would greatly increase radiatieve imbalance; but even after conversion to CO2, the warming would continue. Thus CH4 in PF is crucial. 
%
%Methane clathrates found in terrestrial permafrost (~$<1\%$ of all clathrates), and in marine areas including shallow polar cont shelves ($<1\%$), and deep ocean floor sediments along and at edges of continental slopes ($>98\%$). Fortunately, this later reservoir is insulated by rapid change by the thermal intertia of the oceans, and CH4 released from deep ocean sites is most likely consumed in the sediments and water columns so doesn't reach atmosphere, except as CO2 $10^{1-2}$ years after release (Ruppel). Bad, but not instantaneously bad.
%
%Much more susceptible is the clathrate under permafrost and in shelves. While the risk of rapid significant release\footnote{Mascarelli, A. (2009, March 5). A sleeping giant? [Special Features]. \href{https://doi.org/10.1038/climate.2009.24}{Link}} is possible, even strong surface temperature increases appear to release relatively small volumes of CH4\footnote{Ruppel, C. D., \& Kessler, J. D. (2017). The interaction of climate change and methane hydrates. Reviews of Geophysics, 55(1), 126-168.} This is a rare case where the news isn't all bad!



\section{Perennial (non-seasonal) ice: glaciers, ice caps and the Antarctic and Greenland ice sheets} \label{The Perennial Cryosphere}
\subsection{Introduction}
Humans---despite our marvelous diversity---are all about the same size physically. When we first needed to measure things we used ourselves as rulers. Some of those measures were pretty straight-forward: the foot for example. But other measures-the yard (finger tip to nose) and the inch (a thumb width, try it)---derive from us as well\footnote{see Russ Rowlett's marvelous units of measure site, \href{http://www.ibiblio.org/units/custom.html}{Link}}. Understanding things that don't happen at human scales (either in space or time) is difficult for most of us. So as we begin our study of perennial ice---glaciers and ice sheets, the part of the Cryosphere that lasts more than a few years---we have to remember that some things happen so slowly that fully understanding them takes time. After all, you've probably heard the expression ``glacially slow!'' Most perennial ice moves slowly under its own weight, typically around \SIrange{10}{100}{\metre\per\year}, or about 0.5 inch per hour. These slowly-moving ice masses range from tiny isolated glaciers trapped at the top of mountains to continent-spanning ice sheets of Antarctica and Greenland (Figure \ref{fig:perennial_ice_2}). The ice sheets are the last surviving remnants of ice sheets that over the past few million years repeatedly spanned most of the northern continents, and may again in a few thousand years. The perennial Crysophere is all about long term differnce between mass gain and mass loss, so we begin our discussion there.  \\

But as we saw with the seasonal cryosphere, our real interest in perennial crysosphere is how it interacts with climate. Ice sheets are so large their reflective surfaces change global albedo, their cold surfaces steer atmospheric circulation, and their mass stores 2\% of all surface water. All of these are changing quickly, because this part of the cryosphere is also quickly melting as global temperatures rise. Seasonal snow is too thin to hold much water, and melting sea ice---because it floats---doesn't increase sea levels. But glaciers and ice sheets are largely terrestrial. hen they melt, their water raises sea level. As we saw in Chapter 5 (The Hydrosphere) meting glaciers and ice sheets could raise sea levels as much as \SIrange{1}{2}{\metre} by the end of the century, imperiling the homes of 2.5 to 3.5\% of all humans\footnote{ Rankin, Bill (2016) \href{http://www.radicalcartography.net/index.html?howhigh}{Link}, accessed 30 August 2018. Bias in the underlying data sources concentrates population counts at 1 and 5*n meter intervals; I modeled populations at \SI{0}{\metre} and \SI{1}{\metre} with a second order polynomial fit to observed populations at [2, 3, ...40] m elevation. Residuals were quite low, except for \SI{0}{\metre} and \SI{1}{\metre}, reflecting the counting artifacts in the data.}. This section focuses on the present and near future of glaciers and ice sheets; the fascinating but different story of why and how the Ice Ages happened deserves close inspection, which it receives in Chapter 12, Quaternary Glaciation.  
 
\subsection{Mass balance and movement of ice}
Perennial ice is all about gain \textit{versus} loss. Glaciers form when local snow fall during the winter exceeds melting in the summer. The snow eventually compresses to form ice, which flows downhill---typically\footnote{Marshall, op cit., page 163} at \SIrange{10}{100}{\metre\per\year} (or about half and inch per hour!) under their own weight. At the bottom of the glacier, loss of ice by a variety of mechanisms just matches the forward speed of the glacier and it stops spreading downhill, reaching a point of dynamic equilibrium. If climatic conditions are stable, so too will the glacier's position and shape.\\

\subsubsection{Mass gain} \label{massgain} Snow fall from the atmosphere is essentially the only source of material to the glacier, so glaciers tend to gain mass at their higher elevations. In Chapter 4 The Atmosphere we saw that air temperatures change by $\approx$\SI{-6}{\kelvin\per\kilo\metre}, the upper reaches of glaciers are thus colder and, to first approximation, more likely \emph{accumulate} mass as snow. During the summer, the lower reaches of the glacier will be warmer and hence will lose more snow to \emph{ablation}, the summed melting of ice to water and sublimation of ice to vapor. The line where last winter's snow remains at the end of the summer melt season separates the area where the glacier is gaining mass (the accumulation zone), from the areas of the glacier losing mass (the ablation zone). The altitude separating these two areas---different for different glaciers---is the equilibrium line altitude and represents a sort of ``pivot point'' around which the glacier grows and shrinks. Over the course of a few centuries to millennia with relatively stable climate, the glacier develops an equilibrium profile (Figure \ref{fig:ELA}), with ice flowing in arcuate paths on its way down hill. At the glacier-rock interface, the glacier can slide along the rock, slip against a muddy layer, or even float along a layer of pressurized water. \\
%surface velocities O($10^{2-3}$ m/yr) in AIS/GIS, exclusive of ice streams\\
%surf vel proportional to $H^4*\nabla S^3$, where H is thickness and GradS is surface gradient\\
	%movement accommodated by\\
		%ice creep\\
		%basal sliding of ice over substrate\\
			%where bed is above pressure melting point\\
			%local melting and high p water can float ice and allow fast movement ($10^{3-4}$ m/yr)\\
		%movement of substrate relative to bedrock\\
\subsubsection{Mass loss} Mass gain in glaciers is pretty simple: snow from above. Mass loss is more complicated because there are multiple routes through which water (in any form) can exit glaciers and ice sheets. All glaciers lose ice through \emph{ablation}, the sublimation (see Figure 3.12) of ice to vapor. In areas where air temperatures are relatively warm, such as on the Greenland Ice Sheet, melting of surface ice to form liquid water is an important loss mechanism. The impossibly sapphire water in Figure \ref{fig:glacial_loss} are ponds of glacial melt perched atop the GIS. This water finds its way (panel B), both above and below the ice surface, to the ocean. (One doesn't carry a water bottle when working on glaciers like this; you carry a cup slung to your backpack, dip it into a melt stream drink the coldest and most refreshing water on the planet.) Most glaciers are thick enough\footnote{Marshall, op cit., page 154} to have a layer of melt water at their base, thanks to heat flow from Earth. This melt water (and water from other sources)\footnote{Qiu, J. (2017, November 27). Why slow glaciers can sometimes surge as fast as a speeding train-wiping out people in their path. Retrieved August 30, 2018, from \href{http://www.sciencemag.org/news/2017/11/why-slow-glaciers-can-sometimes-surge-fast-speeding-train-wiping-out-people-their-path}{Link}} can lubricate a glacier, allowing it to surge forward 10 to 100 times faster than its usual speed, fast enough to actually engulf people and livestock. About 60\% of all the mass loss on the GIS is by melting\footnote{AR 5, Ch. 4}, partly due to higher temperatures accompanying Arctic Amplification.\\

While melting is an important loss mechanism, it isn't as splashy as the best-known way glaciers lose mass: by ``calfing'' icebergs. While it rankles a book author to say, no words or still pictures can adequately describe the awesome majesty of icebergs forming at the terminus of a glacier or ice shelve. Fortunately the \textit{Extreme Ice Survey} has recorded such events as part of their survey of global glaciers\footnote{The reader is urged to view a spectualr example of this at https://vimeo.com/5415402.}. Figure \ref{fig:glacial_loss} (panel C) shows recently-formed icebergs (the largest are about one half a kilometer across, or about the size of a large city block. These are \emph{small} compared to the behemoths calved from Antarctic Ice Shelves, as shown in Figure \ref{fig:b15}. The largest iceberg (Figure \ref{fig:b15A} recorded---rather unimaginatively called ``B-15''---was about the area of Connecticut, and drifted through the Southern Ocean for nearly 18 years before it melted into the South Atlantic\footnote{NASA Earth Observatory, \href{https://earthobservatory.nasa.gov/images/92238/end-of-the-journey-for-iceberg-b-15z}{Link}, accessed 31 August 2018}. Because the Antarctic Ice Sheet is too cold to melt, calving is primary mechanism by which Antarctica losses ice\footnote{AR5, op cit}. \\
		%ablation to atmosphere\\
		%surface melting (50\% of ablation in GIS)\\
		%melting into ocean\\
		%basal melting (10\% of ablation in GIS) (Important in AIS)\\
		%dynamic (iceberg calving) (~40-50\% in GIS) (~majority in AIS)\\

\subsection{Glaciers}
Mountain glaciers (Figure \ref{fig:mtn_glacier}) are found, for now, on every continent except Australia,  and even one right on the equator (Navado Cayembe, Ecuador). A crowd-sourced\footnote{Arendt, A., et al., 2012: Randolph Glacier Inventory [v2.0]: A Dataset of Global Glacier Outlines. Global Land Ice Measurements from Space, Boulder Colorado, USA. Digital Media 32 pp. [Available online at: http://www.glims.org/RGI/RGI\_Tech\_Report\_V2.0.pdf]} census of glaciers counted over 170,000 world-wide, including the forlorn Humboldt Glacier of Figure \ref{fig:perennial_ice_1}, and the impressive Aletsch Glacier in Switzerland (Figure \ref{fig:mtn_glacier}. Mountain glaciers form, as we saw above, where long-term mass gain exceeds long-term mass loss. Once large enough, glaciers flow downhill and up temperature until their flow is slower than their melt. Glaciers and climate are intimately connected, so the growth and retreat of glaciers provides an excellent monitor of long-term climate change. If climate cools, glaciers will grow, moving down-hill and inflating as colder conditions push the equilibrium line altitude downhill. In a warming climate, the opposite happens: the glacier melts, retreats up-hill and deflates\footnote{Again words and still pictures are insufficient. The reader is urged to view time-lapse photography of this produced by the EIS, for example https://vimeo.com/5963395}. Glaciers are fierce eroders of rock, and a thick glacier rapidly and effectively carves its own valley with surprising ease, stripping the surface of vegetation, soil, and the smoothing the underlying bedrock. The Aletsch Glacier has been retreating and melting since about 1860 CE\footnote{Anonymous, 2011. \href{http://archive.sciencewatch.com/dr/erf/2011/11feberf/11feberfHolz/}{Link}, accessed 31 August 2018}, which you can see for yourself by looking for the ``trim lines'' the glacier left on its valley. These trim lines record where the glaciers was at its maximum, and are shown with yellow lines for convenience in the Figure \ref{fig:mtn_glacier}. \\

\subsubsection{Mountain glacier melting and sea level rise}
The reason we had to understand mass balance of glaciers before we even looked at a picture of glaciers is straightforward: global warming is melting glaciers all over the globe. As surface temperatures increase, the equilibrium line altitude of many glaciers is actually rising \emph{above the highest point on the glacier.} Once this happens, the glacier is doomed. The most recent Intergovernmental Panel on Climate Change (IPCC) report noted that that at least 600 glaciers world-wide, have disappeared since record keeping began, and many more glaciers are begin to disintegrate. As we seen, glaciers grow or shrink unitl they reach equilibrium, so the observation that most glaceiers global are melting faster with time means that even if warming stiopped now, glaciers would continue to melt. Equilibrium is a long way away. Like the disappearance of sea ice, snow and permafrost, the local and regional implications of these changes are severe. For example, Glacier National Park in Montana is famous for its glaciers, but not for much longer. Only 17\% of the glaciers in the Park present at the end of the $19^{th}$ Century are left, and even those are expected to disappear within a few decades\footnote{USGS (2017) Glacier margin time series (1966, 1998, 2005, 2015) of the named glaciers of Glacier National Park, MT, USA, \href{doi:10.5066/F7P26WB1}{Link}, as quoted in Milman, Oliver, (2017), US Glacier national park losing its glaciers with just 26 of 150 left, The Guardian, 11 May 2017 \href{https://www.theguardian.com/environment/2017/may/11/us-glacier-national-park-is-losing-its-glaciers-with-just-26-of-150-left}{Link} accessed 31 August 2018 }. This will be a catastrophic loss to the local economy. At a local level, glaciers provide a significant portion of drinking water to many communities in the Andes and Himalaya, and the loss of these glaciers threatens the very survival of dependent communities. But at a regional level, glaciers in these areas provide the water used to grow food. Melting glaciers might seem like a good thing---more water---but the glaciers are \emph{shrinking}, so in the long run less water is available for growing crops. The changes in just two river basins in south Asia the Brahmaputra and Indus---threaten the food security of 60 million people\footnote{Immerzeel, W. W., van Beek, L. P. H., \& Bierkens, M. F. P. (2010). Climate Change Will Affect the Asian Water Towers. Science, 328(5984), 1382. https://doi.org/10.1126/science.1183188
}, the equivalent of the populations of California and Texas combined.\\   
\subsection{Ice sheets: Antarctica and Greenland}

\subsubsection{Remnants of the Ice Ages} Both Antarctica and Greenland are covered by extensive ice sheets---continent-spanning glaciers so thick they cover topography and depress their continents into Earth's mantle. Figure \ref{fig:south_pole} shows the tiniest bit of the Antarctic Ice Sheet, viewed from the South Pole. Here the ice is \SI{2800}{\metre} (1.8 miles!) thick\footnote{The average of all BEDMAP 2 (Fretwell, P., Pritchard, H. D., Vaughan, D. G., Bamber, J. L., Barrand, N. E., Bell, R., ... Zirizzotti, A. (2013). Bedmap2: improved ice bed, surface and thickness datasets for Antarctica. The Cryosphere, 7(1), 375-393. https://doi.org/10.5194/tc-7-375-2013; data set version of 5th March 2013, accessed 4 September 2018) point estimates of ice thickness within 20 km of the Pole.}, and is so massive bedrock has sank to sea level. These ice sheets are remnants of once vast sheets of ice that covered Antarctica, much of the Northern Hemisphere and parts of South America. The most recent of these Ice Ages began 34 Mya, and is but the latest episode of a global glaciation, a subject we'll cover in more detail in Chapter 12. Both the Greenland and Antarctica retain ice sheets because they are high, dry and cold, except at their edges. Here we'll examine the contemporary and near future of these great ice sheets, and the implications they hold for contemporary climate change.\\
\subsubsection{AIS}
Like the kid picked last for a team, Antarctica always seems to get regulated to the sidelines of maps, so Figure \ref{fig:rel_size} shows Antarctica, Greenland and North America overlaid and at the same scale. Appreciating the volume of ice in Antarctica requires multiple superlatives. At twice the size of the contiguous United States, Antarctica is 99\% ice covered (Figure \ref{fig:antarc_map}, panel A). The Transantarctic Mountains separate East and West Antarctica, which each have their own ice sheets, with the majority of the ice lying in the Eastern Antarctic Ice Sheet. The Antarctic Peninsula has its own (relatively small) ice sheet, but the northern extremes of the Pennisula lack extensive ice sheets. As you can see from the cross section in Figure \ref{fig:AISXC} and Table \ref{tab:wti}, the average thickness of the AIS is an astonishing \SI{1900}{\metre} (6200 feet), five times the height of the Empire State Building. The deepest ice is more than \SI{4900}{\metre} (3 miles) deep and was most likely deposited in a snow storm more than a million years ago. Driving a snowmobile, sometimes for hours, across the ice sheet (Figure \ref{fig:ice_driving}) while doing research reminds one of traveling across a trackless ocean, so vast are areas covered by ice. Oceans, indeed: melting the AIS would on average raise sea level by \SI{58.3}{\metre} (190 feet); this would raise sea levels to the top of the 10th floor of the Empire State Building (Figure \ref{fig:ESP_SLR}\\
\paragraph{Ice shelves} The edges of Antarctica includes hundreds of floating \emph{ice shelves} (Figures \ref{fig:antarc_map} and \ref{fig:AISXC}), which are the ultimate outlet for glaciers (shown by the purple and red colorings in Panel B of Figure \ref{fig:antarc_map}) draining the ice sheets\footnote{Shepherd, A., Fricker, H. A., \& Farrell, S. L. (2018). Trends and connections across the Antarctic cryosphere. Nature, 558(7709), 223-232. \href{https://doi.org/10.1038/s41586-018-0171-6}{Link}.}. As it flows from outlet glaciers, ice spreads out, thins, and after a millenia for the Ross Ice Shelf\footnote{Fahnestock, M. A., Scambos, T. A., Bindschadler, R. A., \& Kvaran, G. (2000). A millennium of variable ice flow recorded by the Ross Ice Shelf, Antarctica. Journal of Glaciology, 46(155), 652-664.}, reaches the sea. Despite their relatively small size, these floating shelves play an out-sized role in the response of the ice sheet to climate changes, because they hold back, or buttress, the ice shelves from rolling into the oceans.\\

\subsubsection{GIS} Greenland---the world's largest island--- is 80\% covered by ice sheet and glaciers, with an average depth of \SI{1650}{\metre} (5500 feet). The Greenland Ice Sheet, like its larger Antarctic cousin, mantles bedrock topography over most of the island, (see Figure \ref{fig:GISxsection}), sinking the bedrock below sea level in the island's center. The high point of the ice sheet (somewhat unimaginatively named ``Summit'') is near the deepest ice, one of the reasons international teams of glaciologists have drilled and recovered ice cores all the way to bedrock, over 3050 m deep. Like Antarctica, the Greenland Ice Sheet is drained by rapidly-moving outlet glaciers, shown by the red to purple colors in Panel B of Figure \ref{fig:gis_physical}. Greenland's largest glacier, Jakobshavn, is also the world's fastest glacier, in summertime moving as much as \SI{45}{\metre} (150 feet) \textit{per day}\footnote{Joughin, I., Smith, B. E., Shean, D. E., \& Floricioiu, D. (2014). Brief Communication: Further summer speedup of Jakobshavn Isbrae. The Cryosphere, 8(1), 209-214. \href{https://doi.org/10.5194/tc-8-209-2014}{Link}.}. This motion drives frequent iceberg calving at the front of the glacier, which  eventually make their way into the Atlantic Ocean. Until recently, ice loss from the GIS was dominated by calving, as shown in Figure \ref{fig:glacial_loss}; but with Arctic Amplification (see \ref{aa}), ice loss from surface melting has begun to dominate loss in the GIS, which we'll discuss further below. This mass loss is important, as the sea level equivalent of the GIS is an astonishing \SI{7.4}{\metre} (24 feet), equivalent to ~2 stories of the Empire State Building (Figure \ref{fig:esb}).\\

\section{Changes to perennial ice}  \label{Changes to perennial ice}
Despite the local and regional effects of melting perennial ice, we focus in this book on global-scale issues, and that means sea level rise. As you can see in Table \ref{tab:wti}, the sea level equivalents (SLE) of the perennial Cryosphere ranges from \SI{0.41}{\metre} (16 inches) in mountain glaciers to \SI{58.3}{\metre} (190 feet) for the Antarctic Ice Sheet. Between 1990 and 2020 CE, the equivalent of \SI{55}{\milli\metre} (2.1 inches) of sea level rise has melted from the Cryosphere, as shown in Figure \ref{fig:SLR}. The rate of melting is slowly accelerating, as you can see from the slightly upward parabolic sweep to the curves in Figure \ref{fig:SLR}, reflecting increasing melting over time. But it isn't just the volume of water stored in the reservoir that matters, at least on human time scales. Figure \ref{fig:SLR_vs_SLE} shows (in blue) the proportion of SLE in each reservoir, and in orange the proportion of measured sea level rise due to each reservoir\footnote{The SLE proportions are from Table \ref{tab:wti}; the SLR data are for 2020 CE projections from Figure }. While mountian glaciers only hold 1\% of the SLE, they have contributed 50\% of the sea level rise from the entire Cryosphere. This seemingly odd result reflects the more significant and rapid heating of mountain glaciers relative to polar ice sheets in the past few decades.  
As of 2014 CE, glaciers are melting enough to add \SI{1.5}{\milli\metre\per\year} (about 0.06 inches) to global sea level.  By 2020 CE this rate should double, and sea level rise from melting ice will reach an inch per decade. Seems miniscule, until you realize that this tiny amount adds up, year after year, as that new ocean water has no where else to go. Other factors contribute to sea level rise (Chapter 6) but melting ice sheets in Greenland and Antarctica are unique: mass loss in these ice sheets has the potential to raise sea levels suddenly and globally.\\
\subsubsection{Melting of the GIS: Arctic Amplification and Albedo Changes}
The Greenland Ice Sheet loses mass through both calving and melting, but since 2003 CE the rate of melting has increased markedly, and now dominates losses there\footnote{Kjeldsen, K. K., Korsgaard, N. J., Bjork, A. A., Khan, S. A., Box, J. E., Funder, S., ... \& Siggaard-Andersen, M. L. (2015). Spatial and temporal distribution of mass loss from the Greenland Ice Sheet since AD 1900. Nature, 528(7582), 396; Eli Kintisch (2017, February 21). The great Greenland meltdown. Retrieved February 24, 2017, from http://www.sciencemag.org/news/2017/02/great-greenland-meltdown}. Thanks to innovative research techniques, including satellites that precisely measure changes in ice mass and thickness, it is clear that most of Greenland is losing mass, losses equivalent on average to \SI{2.3}{\metre} (7.5 feet) of the Greenland, just since 2003 CE! The combined losses from calving and melting are not uniform: Figure \ref{fig:GIS_melt} shows the loss is concentrated at the ice sheet's margins, particularly near the three large outlet glaciers. Only the highest parts of the ice sheet (shown in white in Figure \ref{fig:GIS_melt} still has a positive mass balance, partly due to orographicly-induced snows on the high part of the sheet.\\
Surface melting is responsible for 70\% of the mass loss in Greenland since 2003 CE, and the cause of this melting is two-fold. The first effect is heating driven by atmospheric circulation changes, ultimately driven by Arctic Amplification and its consequences. Changing jet stream behavior tends to drive relatively brief (a month or two), local, but intense melting episodes as warm air floods into typically cold areas\footnote{Tedesco, M., Mote, T., Fettweis, X., Hanna, E., Jeyaratnam, J., Booth, J. F., ... Briggs, K. (2016). Arctic cut-off high drives the poleward shift of a new Greenland melting record. Nature Communications, 7, 11723. \href{https://doi.org/10.1038/ncomms11723}{Link}.} Of more regional and longer-lasting concern are albedo changes driving ice melt, changes caused by at least three factors\footnote{Kintisch, ibid.}. One is naturally occurring dust distributed through the ice; as ice melts this dark dust concentrates on the surface, decreasing the ice's albedo and increasing melt rates. As the Arctic has warmed and regional wind patterns have shifted, soot (fine-grained and very dark fluffs of carbon from the incomplete combustion of wood) from forest fires has collected on the ice sheet, further reducing the albedo and accelerating melting. Finally, in one of the least expected feedback mechanisms you'll find in this book, the warming ice of Greenland has become the home to curious communities of bacteria and algae that live in and on the ice sheets themselves\footnote{Kintisch, ibid., Tedstone, A, Bamber, J, Cook, JM, Williamson, C, Fettweis, X, Hodson, A \& Tranter, M, 2017, 'Dark ice dynamics of the south-west Greenland Ice Sheet'. Cryosphere, vol 11., pp. 2491-2506}. Pigments in the algae stain the ice, increasing the albedo and inducing additional melting. This additional melting induces more growth of algae, and yet another positive feedback loop enters the Cryosphere, magnifying change. Expect both the area of Greenland Ice Sheet melting, and the total amount of melt, to increase into the future\footnote{AR5, Ch 4.4.2}. \\
\subsection{Antarctic Ice Shelves and Collapse of the West Antarctic Ice Sheet}
Antarctica is ringed by floating ice shelves, where outlet glaciers draining the continent reach the ocean. The outer portions of the shelves float on the ocean, even though they are hundreds of meters thick. Despite their thickness, the floating shelves are eroded by melting at both their upper and lower surfaces. Eventually, cracks propagate through the shelves, cleaving off massive tabular ice bergs(Figure \ref{fig:B15}), which can drift through the southern ocean for years. In recent years, ice shelves in Antarctica, and particularly in the northern reaches of the Antarctic Peninsula, have rapidly retreated, occasionally showing catastrophic and rapid disintegration (Figure \ref{fig:LarsenB}. The ice shelves' disintegration does not in itself contribute to sea level rise. All floating objects displace their own weight of fluid, in this case water in the oceans. So when the ice melts, the water essentially occupies the ``hole'' the ice made in the ocean. But ice shelves act as ``buttress,'' slowing the flow of outlet glaciers into the ocean. Outlet glaciers \emph{do} contribute to sea level rise when they melt. Disappearing ice shelves may allow rapid and catastrophic increases in sea level.\\

\subsubsection{Marine ice sheet instability} Figure \ref{fig:buttress} shows an outlet glacier and its fringing ice shelf. The shelf melts from both above and below, and both of these sources of melting are accelerating because of climate change. Air temperatures in the Antarctic Peninsula have increased four to six times faster than the global average\footnote{AR 5, Ch. 4}, which leads to increased surface melting. This melted water runs into cracks in the shelf (Figure \ref{LarsenB}, panel A), and wedges the ice apart, hastening its disintegration\footnote{Shepherd, op cit.}. But climate change is also warming the oceans, which melt the ice shelves from below (Figure \ref{fig:buttress}). The relatively warm and fresh water created by this melting rises along the edge of the shelf, contributing to more melting. As this process continues the grounding line---where the ice shelf and underlying bedrock last touch---retreats inland. Unfortunately, the weight of the grounded ice depresses the bedrock, which gets \emph{deeper} inland, allowing the melting and retreating to accelerate. As the grounding line moves inland, the glacier gets thicker and moves faster, accelerating ice loss from the outlet glaciers. This process is at least partly responsible for the loss of one fifth of all ice shelves in the Antarctica since 1950 CE\footnote{Shepherd et al. op cit.}. Loss of ice shelves has allowed some outlet glaciers in Antarctica to surge forward, accelerating loss of Antarctic Ice Sheet and increasing sea levels. This marine ice sheet instability (MISI) hypothesis\footnote{Bamber, J. L., Riva, R. E. M., Vermeersen, B. L. A., \& LeBrocq, A. M. (2009). Reassessment of the Potential Sea-Level Rise from a Collapse of the West Antarctic Ice Sheet. Science, 324(5929), 901-903. \href{https://doi.org/10.1126/science.1169335}{Link}} has profound implications for sea level rise this century.\\

So far, most of the photogenic ice shelf losses has been on the Antarctic Peninsula, which (fortunately) has a SLE\footnote{Pritchard, H. D., \& Vaughan, D. G. (2007). Widespread acceleration of tidewater glaciers on the Antarctic Peninsula. Journal of Geophysical Research: Earth Surface, 112(F3).} of only \SI{0.24}{\metre} (9.5 inches), far less than mountain glaciers. In West Antarctica, on the other hand,  about half the ice is susceptible to MISI, and all that ice is buttressed by the three large and dozens of smaller ice shelves, the later of which are particularly susceptible to disintegration because of their exposed location. These shelves hold back ice with a SLE of \SI{3.4}{\metre} (11 feet), enough to flood the homes of 230 million people---4\% of all people on the planet. The IPCC concluded that there is medium confidence that West Antarctic Ice Sheet is losing mass\footnote{AR 5, Ch. 4, pg. 357}, but the chances that it will catastrophically disintegrate in the next 200 years is extremely unlikely---perhaps 5\%. This is a classic example of the ethical and practical difficulties associated with climate change. Recall that we define risk as the product of consequence and probability. An outcome with zero consequence clearly carries little risk: an example might be clicking on the wrong link on a web page. In the same way, an event with zero probability also has no risk associated with it. But WAIS collapse and attendant sea level rise are different. The probability is low, but the consequences would clearly be catastrophic to the 4\% of humans whose lives were destroyed by flooding. Oc course the lowest elevation areas are near the ocean, which in developed countries are typically occupied by expensive dwellings and residents who want to live at the ocean's edge. How would you explain to such a resident the risk of WAIS collapse and the flooding of their home?\\
\subsubsection{Gravitational bulges and identifying glaciers threatening \emph{your} hometown} One quirk of ice loss from either Antarctica or Greenland is un-even way that sea levels rise across the globe. All that ice in the AIS and GIS has substantial mass, and so exerts gravitational force on its surroundings. This force is enough to raise local sea levels by as much as a few meters: both Antarctica and Greenland are surrounded by bulges of ocean water attracted by their ice sheets. If the ice were to melt and the water flow away, so too would the ocean bulges. Ironically, the disappearance of the bulges just about cancels out the  sea level rise from the melting, which means that sea level rise is amplified in areas far from the melting ice. For example, were the entire WAIS to melt, the resulting sea level increase would \emph{average} \SI{3.3}{\metre}, but would be negative close to West Antarctica and more than 25\% greater along the US coast, which would see \SI{4}{\metre} of sea level increases! A remarkable study released in 2017 CE allows users to enter a sea-side city and identify precisely which glaciers present the greatest risk to flooding. 
      


% shown in Figure \ref{fig:antarc_map}, most of the continen %Mass balance
	%Gain through precip via atmosphere\\
	%Loss through \\
		%ablation to atmosphere\\
		%surface melting (50\% of ablation in GIS)\\
		%melting into ocean\\
		%basal melting (10\% of ablation in GIS) (Important in AIS)\\
		%dynamic (iceberg calving) (~40-50\% in GIS) (~majority in AIS)\\
%AR5 says ``Ice loss from Greenland is partitioned in approximately similar amounts between surface melt and outlet glacier discharge (medium confidence), and both components have increased (high confidence). The area subject to summer melt has increased over the last two decades (high confidence). {4.4.2}''

\section{Cryosphere over Geologic Time}  \label{Snow-ball Earth}
[[placeholder]]
\subsection{Proterozoic: Snowball Earths}
[[placeholder]]
\subsection{Quaternary Ice Ages}
[[placeholder]]

\section{Recent and future changes to Cryosphere}  \label{Our Rapidly Changing Cryosphere}
\subsection{Rivers and lakes}
\subsection{Snow???}
``The rate of SCE loss during June from 1979 to 2011 of 17.8\% per decade is even faster than the loss rate of Arctic SIE (= Derksen, C. \& Brown, R. Spring snow cover extent reductions in the 2008-2012 period exceeding climate model projections. Geophys. Res. Lett. 39, L19504 (2012), as quoted in Tang, Q., Zhang, X., \& Francis, J. A. (2014). Extreme summer weather in northern mid-latitudes linked to a vanishing cryosphere. Nature Climate Change, 4(1), 45.)''

From Tang, Q., Zhang, X., \& Francis, J. A. (2014) 
\begin{quotation}
	Fifth, the response to the combined SIC/SCE loss suggests that the zonal jet-stream winds are weakened and the jet is shifted northward. The large-scale waves are amplified, which tends to favour more persistent weather systems and a higher likelihood of summer weather extremes. Although there is much still to learn about the interactions between a rapidly changing Arctic and large-scale circulation patterns, this study builds on earlier work and provides further evidence linking snow and ice loss in the Arctic with summer extreme weather in mid-latitudes. As greenhouse gases continue to accumulate in the atmosphere and all forms of Arctic ice continue to disappear, we expect to see further increases in summer heat extremes in the major population centres across much of North America and Eurasia where billions of people will be affected.
\end{quotation}


\subsection{Permafrost}
\subsection{Sea Ice}
Notz and Stroeve CO2 emission and sea ice decline\\
Changes to extreme wx frequency\\
\subsection{Mountain Glaciers}
\subsection{AIS/GIS}
from Kusahara, K. (2016). Cryosphere: Warming ocean erodes ice sheets. Nature Climate Change, 6(1), 22. \begin{quotation}
	The behaviour of ice shelves is very important for maintaining the ice sheets. Changes in the shape and size of the shelf, whether due to collapse or increased basal melt, can weaken the maintaining mechanism. Much of the West Antarctic Ice Sheet exists on bedrock that is below sea level, creating the possibility of marine-ice-sheet instability. This instability comes from the combination of the retrograde bed slope to the centre of the ice sheets and the decreasing ocean freezing point with depth. Simply, it means that once the grounding line has retreated, basal melting at the grounding line (which is moving deeper) becomes more active, thus leading to further retreat of the grounding line and ultimately massive loss of the ice sheet.
\end{quotation}

\section{Figures and Tables}   \label{Cryosphere_Figures}

\begin{figure}[p]
\centering
\includegraphics[width=6 in]{barentsburgcoalpowerplant}%
\caption{The opening photograph illustrates the the central theme of this chapter: Earth's icy Cryosphere (represented by the rapidly melting Gr\o nfjord Glacier in the background) is rapidly melting because of anthropogenic climate change, fueled by $CO_2$ emissions from burning  fuels (represented by the Barentsburg power plant in Svalbard). Temperature change in the Arctic is two to three times the global average, a pattern called ``Arctic Amplification. Photograph by Thomas Nilsen, The Barents Observer. I DO NOT YET HAVE PERMISSION TO USE THIS IMAGE. Original caption: ``In times of climate changes, the view couldn't be more contradictory; a coalmine with a smoky coal power plant in front of a melting glacier. In recent years, the Arctic, and especially Svalbard, has warmed twice as much as the rest of the planet. Presumably owing to warming, most of the glacier retreat rates on Svalbard have increased several folds in recent decades. That includes the Gronfjord glacier just south of Barentsburg. The coal plant in Barentsburg consumes about 30,000 tons of coal annually.  Photo: Thomas Nilsen'' https://thebarentsobserver.com/en/node/4314?utm\_source=EHN\&utm\_campaign=ee2e6d6857-RSS\_EMAIL\_CAMPAIGN\&utm\_medium=email\&utm\_term=0\_8573f35474-ee2e6d6857-99389405} 
\label{fig:Barentsburg}
\end{figure}


\begin{figure}[p]
\centering
\includegraphics[width=6 in]{kebenkaises}%
\caption{The rapidly variable cryosphere. Panel A is a picture of Sweden's high point, the lovely Kebenkaises mountain. At the time the photograph was taken, the southern peak (white arrow) was the high point, at roughly \SI{2100}{\metre} above sea level. The photograph is by Alexandar Vujadinovic, available at \href{https://upload.wikimedia.org/wikipedia/commons/thumb/4/45/Kebnekaise_viewed_from_Tarfala_valley_-_narrower_crop.jpg/1024px-Kebnekaise_viewed_from_Tarfala_valley_-_narrower_crop.jpg}{Link} and is under CC 4. Panel B shows the difference between daily and average maximum temperatures for 02 July 2018, one day of a record heat wave suffered by Europe in the summer of 2018. The yellow diamond marks the location of Kebenkaises, where temperatures were about \SI{10}{\celsius} higher than normal. The image is courtesy of Climate Reanalyzer at the University of Maine (\href{https://climatereanalyzer.org/reanalysis/daily_maps/}{Link}). In Panel C, a scientist uses a differential GPS to measure the height of the southern peak, which melted by more than \SI{4}{\metre} in July of 2018 alone. The rocky northern peak (red arrow in Panel A) became the high point as the southern peak wasted away. This figure and the details of the caption are from an article, in Swedish, \href{http://www.mynewsdesk.com/se/su/pressreleases/kebnekaises-sydtopp-blir-laegre-aen-nordtoppen-i-augusti-2611533}{here}. I do not have permission yet to use Panel C.} 
\label{fig:kebenkaises}
\end{figure} 


\begin{figure}[p]
\centering
\includegraphics[width=4 in]{cryo_albedo_feedback}%
\caption{The cryosphere-albedo-temperature feedback (upper portion with dashed outline) is a positive feedback system exerting great leverage on Earth's climate. Through connections with ocean temperature (lower portion) this system also intersects the carbon cycle, which ends up twiddling the greenhouse effect thermostat and further extending the feedbacks leverage on climate. Begin with an increase, for whatever reason, in the area covered by highly reflective snow or sea ice. This increases albedo (note the solid green ``gain'' between the two boxes, indicating they change in the same sense), which \emph{decreases} (red, dashed gain) absorbed short wave energy from the Sun. In turn this reduces Earth's equilibrium temperature $T_{eq}$, then the average surface temperture $T_{surf}$, which increases the stability of snow and ice and so starts the cycle anew. Over time scales of a few hundreds to thousands of years---quite short geologically---the deep oceans begin to cool as well, allowing more $CO_2$ to dissolve into them. This $CO_2$ comes from the atmosphere, so the greenhouse effect lessens, leading to further reductions in $T_{surf}$ and stabilizing the cryosphere even more. These coupled systems play a crucial role in strengthening and terminating recent Ice Ages, and (in the Proterozoic) initiating ``Snowball Earth'' periods.} 
\label{fig:cryo_albedo_feedback}
\end{figure}



\begin{figure}[p]
\centering
\includegraphics[width=6 in]{cryo_concept_map}%
\caption{As we saw in Figure \ref{fig:jonas}, a single snowstorm can thrust 100 million people into the Cryosphere overnight. Despite this, most people are less familiar with the Cryosphere than any other major system on Earth. The figure maps out the way a changing Cryosphere (in the center) affects other parts of Earth, including the Anthroposphere (shown in blue \textit{italic} type). The red arrows indicate positive feedbacks which amplify changes to the Cryosphere. The power subtle power of the Cryosphere is evidenced best two end-members of its effects: changes to winter tourism at one end, and changes to Earth's very shape on the other.} 
\label{fig:cryo_concept_map}
\end{figure}

\begin{figure}%[ht]
\centering
\includegraphics[width=5 in]{jonas}%
\caption{An intense winter storm in late January of 2016 brought the Cryosphere to a third of the people in the United States, and left so much snow it was visible from space. This image from NASA's Aqua satellite was taken mid-day on January 24, 2016. Although snow is visible over much of the image, the snow from that storm fell (inset image) in a band from Arkansas to Massachusetts. The Cryosphere---the domain of frozen water---is stable only where temperatures are below the freezing point of water. Daily and seasonal variations in temperature mean the Cryosphere is exceptionally volatile, more so than any of the Earth's other spheres, even the Biosphere. Long term changes in Earth's surface temperature---both natural and man-made---can push the Earth into ``Hot House'' eras marked by the near absence of a Cryosphere; or by''Snowball'' stages, with glaciers covering the continents even to sea level at the equator.} 
\label{fig:jonas}
\end{figure} 

\begin{figure}[p]
	\centering
  \includegraphics[width=4 in]{wheres_the_cryo}%
	\caption{The Cryosphere's components, by volume, roughly equivalent to mass. When examined through this lens, the Antarctic and Greenland Ice Sheets dominate the Cryosphere, storing roughly 98\% of all Earth's ice. Glaciers and sea ice, although they cover a wide area of each pole, are so thin they contribute little mass to the system. Permafrost and snow aren't shown on the diagram, as the former is rather difficult to quantify, and the latter is too thin and temporary to matter. The Antarctic Ice Sheet (AIS) has three distinct parts, as shown on the upper inset diagram: East (the largest, darkest blue), West (medium blue), and Peninsula (the smallest, shown by the black line). The continent is surrounded by extensive winter-time sea ice (inset). The Greenland Ice Sheet---the last vestige of the extensive ice sheets of the most recent Ice Age---is continuing to melt, increasing relative sea levels. Arctic sea ice cover is shown in the inset at roughly its maximum extent, although this has been rapidly decreasing for the past two decades. Melting of this floating ice does \emph{not} increase sea level. Roughly, each 1\% of ice shown in the figure is equivalent to \SI{0.7}{\milli\metre} of sea level rise, roughly the annual amount of the AIS and GIS annually meting into the oceans. The inset figures are courtesy of NASA,\href{https://svs.gsfc.nasa.gov/11703}{Link}, The Arctic and the Antarctic Respond in Opposite Ways Released on October 8, 2014, accessed 20 August 2018. }
	\label{fig:wheres_the_cryo}
\end{figure}

\begin{figure}[p]
	\centering
  \includegraphics[width=3 in]{snowfall}%
	\caption{Even in as static a medium as a woodcut, the gentle sounds and sights of snow falling resonate with anyone who has experienced it first-hand. Seiji Sano (Japan, b 1959) Snow Falling Softly 2004 woodblock print. I do not have permission yet to use this figure.}
	\label{fig:snowfall}
\end{figure}

\begin{figure}[p]
	\centering
  \includegraphics[width=5 in]{orographic.png}%
	\caption{This splendid view of California from NASA's Modis instrument shows the power of orographic uplift in creating snow. The fertile, green Central Valley (in the center) is walled on the right by the snow-capped Sierra Nevada mountains. Storms move from the west to east here, and as air masses rise up to clear the Sierra Nevada range, they cool and drop their water as snow. The inset shows the topography (in meters) along the base of the insert; note that the deserts to the east are as high as much of the snow-covered Sierra's, but lack snow, due to the ``snow shadow'' effect of the mountains. This image is courtesy of NASA/Modis \href{https://www.ncdc.noaa.gov/sites/default/files/Sierra-Nevada-snowpack-comparison-web-quality.jpg}{Link}.  }
	\label{fig:orographic}
\end{figure}

\begin{figure}%[p]
	\centering
  \includegraphics[width=5 in]{seasonalsnow}%
	\caption{Snow cover in the Northern Hemisphere in (left) wintertime maximum and (right) summer-time minimum. Gray areas are snow covered. As the globe warms, the area covered by snow is decreasing, particularly at lower latitudes and elevations. Overall reductions in average snow cover have decreased by about 7\% since the 1920s CE. This image is courtesy of the National Snow and Ice Data Center,  \href{https://nsidc.org/cryosphere/snow/science/where.html}{Link}, accessed 21 August 2018.}
	\label{fig:seasonalsnow}
\end{figure}

\begin{figure}%[p]
	\centering
  \includegraphics[width=3 in]{NHSCEmaxima}%
	\caption{The limits of snow cover in the Northern Hemisphere. Areas in dark blue were \emph{always} covered by snow in every week in every January from 1967-2015---they are permanently in the Cryosphere. Areas in white were \emph{never} covered by snow in any week of January in that period, so were never part of the Cryosphere. Areas in light blue are those on the edge, part of the Cryosphere that is at risk of  disappearing in a globally-warming world. Compare this to Figure \ref{fig:sce} Figure 3 from Kunkel et al. Op cit.}
	\label{fig:scemax}
\end{figure}

\begin{figure}%[p]
	\centering
  \includegraphics[width=3 in]{NHSCEchanges}%
	\caption{Changes to snow covered areas in the past 40 years. Brown areas lost snow cover; blue areas gained. In Europe (on the right) southerly areas lost snow, as we predicted in Figure \ref{fig:scemax} and section \ref{snow_talk}. In North America (on the let) the pattern is more complicated, with minor gains (as we predicted) in the Central Plains and heavy losses (as we didn't predict!) in the Rocky Mountains. This is probably due to the severe drought in the Western United States during this time. Climate is complicated. Figure 1 from Kunkel et al. Op cit.}
	\label{fig:sce}
\end{figure}


\begin{figure}[p]
	\centering
  \includegraphics[width=5 in]{seaice_anncycle}%
	\caption{Upper pane, the average annual cycle of sea ice extent in the Arctic, as measured by satellites since 1979 CE. At the end of the melting season (September) sea ice grows as the sunlight fails and air temperatures plummet. Sea ice reaches a maximum extent in February or March, and then slowly begins to melt as the sunlight returns to the far polar reaches. Blue shading shows the ranges over which 90\% (light blue area) and 50\% (darker blue) of all data lay. The dark blue line shows the median. The red and orange lines show the trend for 2018 and 2012 CE, the all time lowest ice recorded. The inset maps show the distribution of sea ice for an typical year. Lower pane, the annual cycles of both Northern Hemisphere (blue) and Southern Hemisphere (orange) sea ice. Note the larger annual variation in the Southern Hemisphere, and the reversed seasons. Although trends are hard to see int this figure, a slow decrease in both maximum and minimum sea ice in the Arctic is visible. The inset maps show the distribution of Southern Hemisphere sea ice for an typical year at both September maximum and February Minimum. Data from Fetterer, F., K. Knowles, W. Meier, M. Savoie, and A. K. Windnagel. 2017, updated daily. Sea Ice Index, Version 3. Boulder, Colorado USA. NSIDC: National Snow and Ice Data Center. doi: https://doi.org/10.7265/N5K072F8, Accessed 22 August 2018, at https://nsidc.org/data/seaice\_index.}
	\label{fig:seaice_anncycle}
\end{figure}

\begin{figure}[p]
	\centering
  \includegraphics[width=6 in]{arcticamp}%
	\caption{One illustrative example of Arctic Amplification: warming at the North Pole. The main figure shows the temperature changes (relative to 1959-1980) at the North Pole (solid blue lines) and for the world as a whole (dashed orange lines). Since 1980 CE, the North Pole has warmed 3 times faster than the world as a whole. The inset shows predicted temperature changes over a likely range of GHG emissions; by 2100 CE may warm by \SIrange{5}{10}{\celsius} ($9\ to\ 18^{\circ}$F) above current already warm temperatures. [Data for the North Pole is compliments of Zeke Hausfather, Berkeley Earth, personal communication 23 August 2018, ``Its [sic] observational data from Berkeley Earth and the CMIP5 multi-model mean for each RCP. Anomalies are plotted with respect to a 1951-1980 mean, and models have been aligned to observations over the last 20 years (1999-2018).'' Global data are the GISS data, wrt a 1951-1980 mean. Predicted ranges are CMIP5 means forced by RCP85 (upper) and RCP 60 (lower).The global results are compliments of the (amazing) KNMI Climate Explorer, https://climexp.knmi.nl/plot\_atlas\_form.py.]}
	\label{fig:aa}
\end{figure}



\begin{figure}[p]
	\centering
  \includegraphics[width=5 in]{USSHonolulu_polar_bears}%
	\caption{In the left pane, three Polar bears approach the starboard bow of a US Navy submarine while surfaced \SI{520}{\kilo\metre} (280 miles; I assume these are nautical miles) from the North Pole in October of 2003 CE. According to the crew members, the bears investigated the boat for almost 2 hours before leaving. The ice here is relatively thin, as you can see from the pieces stranded on the submarine's hull. This U. S. Navy photo by Chief Yeoman Alphonso Braggs (number 031000-N-XXXXB-001) was twice chosen as a Wikipedia \textit{Picture of the Day}.}
	\label{fig:sub_polar}
\end{figure}


\begin{figure}[p]
	\centering
  \includegraphics[width=5 in]{swimsanta1}%
	\caption{Upper pane: Annual monthly minimum sea ice extent in the Arctic (solid blue) and Antarctic (orange dashed line). Extent in the Arctic shows, despite substantial year-to-year variations due to natural variability, a clear and accelerating decline, averaging about 1.4\% loss \emph{per year} since 1979. Sea ice extent in Antarctica is more stable, and while it may be growing slowly (AR5, Chapter 4, page 333ff.) recent wide swings in extent suggest future changes may be imminent. Lower pane: Determining when the Arctic will be ice free at the end of the summer melt season (not all year!) begin by mathematically modeling the recorded losses as a function of time. The natural year-to-year variations in the observed minimium areas lead to uncertainty in the results, expressed by the best estimate (solid red line) and confidence interval (red dashed lnes). The best estimate predicts an ice-free Arctic ocean---for the first time in the last 100,000 years---by 2040 CE, and a 5\% chance by 2032 CE. Even the most distant likely date-2066-is in the lifetime of just about every reader of this book.    }
	\label{fig:min_ice}
\end{figure}

\begin{figure}[p]
	\centering
  \includegraphics[width=6 in]{unprecedented}%
	\caption{This image from NASA's Modis instrument shows the very northern part of Greenland (red rectangle on inset map shows the location) and adjacent Arctic Ocean on 19 August 2018. Sea ice covers the majority of the Arctic Ocean, here glimpsed between clouds tinted pink for clarity. Against the Greenland Coast though the sea ice is broken in spots, and vast areas of the open ocean are ice free. This is typically the refuge of the thickest sea ice in the Arctic Ocean, generally four times as thick as normal sea ice, even in summer. But melting induced by abnormally warm weather---made more likely by climate change and Arctic Amplification---allowed the ice to be broken up by winds. This area was once expected to be the ``last refuge'' of ice in the Arctic. Image courtesy of NASA Worldview application (https://worldview.earthdata.nasa.gov/) operated by the NASA/Goddard Space Flight Center Earth Science Data and Information System (ESDIS) project, accessed 21 August 2018}
	\label{fig:breakup}
\end{figure}


\begin{figure}[p]
	\centering
  \includegraphics[width=3 in]{permafrost_distribution_in_the_arctic}%
	\caption{Permafrost forms where soil temperatures remain below freezing points for years. Like snow cover, the stability of permafrost changes with distance from the pole, as shown here by continuous areas (in darkest purple) to areas with only isolated pockets (lightest purple). Except for a few places in South America and Antarctica, permafrost is rare in the Southern Hemisphere. Figure courtesy of he National now and Ice Data Center. https://nsidc.org/sites/nsidc.org/files/images//permafrost\_distribution\_in\_the\_arctic.jpg }
	\label{fig:permafrost}
\end{figure}

\begin{figure}[p]
	\centering
  \includegraphics[width=3 in]{Burning_Gas_Hydrates}%
	\caption{Clathrates are ``cages'' of ice wrapped around gas molecules. Gas molecules are normally spread relatively far apart, but dense ice allows clathrates to concentrate gas by almost a factor of 200. So much so that clathrates \emph{burn} when ignited, as shown here. The whitish chunks of ice the bottom (laying on a wire mesh) release gas as they warm, supplying the purplish-red flame with new methane. THis photograph is courtesy of J. Pinkston and L. Stern (USGS). https://www.usgs.gov/media/images/gas-hydrates-burning.}
	\label{fig:clath_fire}
\end{figure}

\begin{figure}[p]
	\centering
  \includegraphics[width=5 in]{perennial_ice_2}%
	\caption{Panel A: A panoramic image of eastern Greenland beuatifully illustrates the various types of perennial ice, including glaciers, ice caps (ice isolated from the main ice sheet) and the massive Greenland Ice Sheet, at the horizon. Credit: Frank Paul, University of Zurich https://www.flickr.com/photos/gsfc/8741348411/in/photostream/. Panel B shows the very last remnant of the very last glacier in Venuzuela, \SI{4900}{\metre} high in the Andes. ``In 1910, glaciers spanned an area of at least 10 square kilometers (4 square miles) in the mountainous region of northwestern Venezuela. As of November, 2015, when this image was taken, less than one percent of that glaciated area remains, and all of it is locked up in one glacier. The ongoing retreat of Humboldt Glacier-Venezuela's last patch of perennial ice-means that the country could soon be glacier-free. Carsten Braun, a scientist at Westfield State University in Massachusetts who has surveyed the glacier predicted in 2015 that the glacier would disappear in the next 10 to 20 years.'' The image is courtesy of NASA Earth Observatory images by Joshua Stevens, using Landsat data from the U.S. Geological Survey and topographic data from the Shuttle Radar Topography Mission (SRTM). The quoted portion of the caption is by Kathryn Hansen of NASA. https://earthobservatory.nasa.gov/images/92659/last-glacier-standing-in-venezuela}
	\label{fig:perennial_ice_2}
\end{figure}

\begin{figure}[p]
	\centering
  \includegraphics[width=5 in]{ELA}%
	\caption{Upper panel: The spectacular Ruth Glacier, in Denali National Part, Alaska (inset), drains the southeast side of Denali, the high point of North America. The brown streaks along the glaciers are lateral morraines, piles of dirt, rock and ruble dumped onto the glacier's surface upstream. Credit: Google Earth. Lower panel: Glaciers and ice sheets grow when winter accumulation of snow fall exceeds summertime loss by ablation. After many centuries to millenia, a glacier develops an equilibrium shape, with the accumulation area that ends at the equilibrium line altitude. The profile is not to scale.  }
	\label{fig:ELA}
\end{figure}

\begin{figure}[p]
	\centering
  \includegraphics[width=6 in]{meltandcalve}%
	\caption{Glaciers lose mass by melting, sublimation and calving of icebergs, two of which are illustrated in this impressive image from NASA's \textit{Landsat} satellite. The Greenland Ice Cap (right) slides downhill into the Baffin Bay (left) through hundreds of outlet glaciers middle). A line of sapphire-blue melt pools dots the transition between continuous ice and the outlet glaciers. This water collects in pools and then (panel B; the colors have been stretched to show the abundance of melt channels) flows through channels either off or \textbf{into} the glacier. There it helps lubricate the outlet glacier's flow, accelerating the calving of icebergs, shown in panel C. Robert Simmon made the original image, which is courtesy of NASA's Earth Observatory, https://earthobservatory.nasa.gov/images/7569  }
	\label{fig:glacial_loss}
\end{figure}

\begin{figure}[p]
	\centering
  \includegraphics[width=5 in]{B15A}%
	\caption{Caption from NSF/USAP: ``The northern edge of the giant iceberg, B-15A, in the Ross Sea, Antarctica. Iceberg B-15A is a fragment of a \emph{much larger iceberg} (B-15) that broke away from the Ross Ice Shelf in March 2000. Scientists believe that the enormous piece of ice broke away as part of a long-term natural cycle (every 50-to-100 years, or so) in which the shelf---which is roughly the size of Texas---sheds pieces much as human fingernails grow and break off. In 2005 ocean currents took B-15A slowly past the Drygalski ice tongue; the collision broke off the tip of Drygalski in mid-April. Iceberg B-15A sailed on along the coast leaving McMurdo Sound until it ran aground off Cape Adare in Victoria Land (a region of Antarctica lying south of New Zealand), where it broke into several smaller pieces on Oct. 27 and 28, 2005. On November 21, 2006 several large pieces were seen just 60 kilometres (37 mi) off the coast of Timaru, New Zealand, the largest measuring about 1.8 kilometres (1.1 mi) wide and 120 feet (37 m) high. Photo by and courtesy of Josh Landis and the NSF/USAP. https://photolibrary.usap.gov/PhotoDetails.aspx?filename=EDGE02.JPG.}  
	\label{fig:B15}
\end{figure}


\begin{figure}[p]
	\centering
  \includegraphics[width=6 in]{Aletsch_Glacier.png}%
	\caption{The Aletsch Glacier in Switzerland is the largest valley glacier in the Alps and it has been losing mass since the mid-19th century. Its volume loss since the middle of the 19th century is well-visible from the trimlines, the line of highest advance of hte glacier, marked in yellow. A new study using data from two NASA satellites found that glaciers like this one lost an average of 571 trillion pounds of ice per year from 2003 to 2009, which contributed to about 30 percent of the total observed global sea level rise during the same period. Credit: Frank Paul, University of Zurich. Credit: Frank Paul, University of Zurich https://www.flickr.com/photos/gsfc/8741348531/in/photostream/}
	\label{fig:mtn_glacier}
\end{figure}

\begin{figure}[p]
	\centering
  \includegraphics[width=6 in]{south_pole}%
	\caption{The tiniest portion of the Antarctic Ice Sheet, seen from the (ceremonia }
	\label{fig:south_pole}
\end{figure}


\begin{figure}[p]
	\centering
  \includegraphics[width=3 in]{relative_sizes_GIS_AIS.png}%
	\caption{Antarctica (in yellow), Greenland (in green, of course) and North America (gray) showing their correct relative sizes, which most maps can't do correctly. Antarctica is nearly twice the area of the lower 48 states; Greenland is about the area of the Western US. Nearly all (99\%) of Antarctica is ice covered, with an average thickness of about \SI{1900}{\metre} (6200 feet), five times the height of the Empire State Building. Greenland---the world's largest island--- is 80\% covered by ice sheet and glaciers, with an average depth of \SI{1650}{\metre} (5500 feet), three times the height of the Empire State Building.}  
	\label{fig:rel_size}
\end{figure}

\begin{sidewaysfigure}[htp]
	\centering
  \includegraphics[width=9 in]{antarc_physical.png}%
	\caption{Antarctica and its perennial ice. Panel A (left) shows the physical features of the continent, with ice in white, exposed rock in brown, floating ice shelves in light blue, and principle bases marked by red squares. Note that the East and West Antarctic Ice Sheets are largely separated by the Transantarctic Mountains, while the ice sheet of the Antarctic Peninsula sticks off to the northeast. Figure is courtesy of Wikipedia, (https://en.wikipedia.org/wiki/Geography\_of\_Antarctica\#/media/File:Antarctica.svg) accessed 7 September 2018. Panel B shows the ice sheet's motion, with speeds indicated by the color scale in the upper left. Black lines indicate flow divides; places where ice flows in opposite directions. Note how the ice sheets drain through outlet glaciers (in reds and purples) to the peripheral floating ice shelves. The red (Panel A) and white (Panel B) show the cross section line in Figure \ref{fig:AISXC}. Panel B is courtesy of NASA's Goddard Space Flight Center Scientific Visualization Studio, https://svs.gsfc.nasa.gov/3849, accessed 7 September 2018. }  
	\label{fig:antarc_map}
\end{sidewaysfigure}


\begin{figure}[p]
	\centering
  \includegraphics[width=7 in]{antarcticaxsec.png}%
	\caption{A cross section of Antarctica, showing the ice sheet (in lighter blue), bedrock (brown) and ocean (darker blue). The cross section (inset map) starts near the McMurdo Base, passes through Dome Argus and Dome Fuji, and ends at Queen Maud Land. Ice thickness (in meters) is shown by dashed blue lines. Note how the smoothed-surfaced ice sheet covers the bedrock topography, in some places by over \SI{3100}{\metre}, except for isolated out cropings in the Antarctic (on the left) and Orvin (on the right) Mountains.The ice sheet flows downhill, spreading from the high point toward the margins of the continent. Along the continent's margin are a number of floating ice shelves. The Ross Ice Shelf (a few hundreds of meters thick) remains the largest, but the Fimbul (which means ``Giant Ice''[Really! see Fimbulisen (the giant ice) https://geonames.usgs.gov/apex/f?p=gnispq:5:0::NO::P5\_ANTAR\_ID:4905]) Ice Shelf is probably the best-named. Ice shelves are particularity susceptible to break-up die to warming oceans, as shown in Figure \ref{fig:buttress}. Ice surface and bedrock elevations are from BEDMAP2, Fretwell, P., Pritchard, H. D., Vaughan, D. G., Bamber, J. L., Barrand, N. E., Bell, R., ... Zirizzotti, A. (2013). Bedmap2: improved ice bed, surface and thickness datasets for Antarctica. The Cryosphere, 7(1), 375-393. \href{https://doi.org/10.5194/tc-7-375-2013}{Link}, data of 5th March 2013, accessed 4 September 2018.}  
	\label{fig:AISXC}
\end{figure}

\begin{figure}[p]
	\centering
  \includegraphics[height=5 in]{ESB_SLR}%
	\caption{The sea level equivalent of the Cryosphere and its constituents is a handy way of conceptualizing the amount of water stored in ice. If all ice were melted and allowed to run to the oceans, sea levels would rise an average of \SI{66}{\metre}, or 220 feet! Were this to happen---and melting of \emph{all} ice in the Cryosphere is exceptionally unlikely on human timescales---sea level would rise high enough to cover the first 13 floors of the Empire State Building, as shown above. The contribution from melting glaciers, the Greenland Ice Sheet and the Antarctic Ice Sheets are shown separately above, from the top to bottom.}  
	\label{fig:ESB}
\end{figure}


\begin{figure}[p]
	\centering
  \includegraphics[width=7 in]{GISxsection.png}%
	\caption{A cross section of Greenland, showing the ice sheet (in lighter blue), bedrock (brown) and ocean (darker blue). The cross section runs from the northwest corner of the island, to ``Summit'' (the high point) and then to the southwestern corner of the island. Ice thickness (in meters) is shown by dashed blue lines. The ice sheet mantles topography thoroughly, leaving rock exposed only at the edges. The approximate location of deep ice cores drilled through the ice sheet is shown by the black line labeled ``GISP.'' This ice core has provided a rich history of the ice ages back to ~800 ky. Ice surface and bedrock elevations are from Bamber, J. L., Griggs, J. A., Hurkmans, R. T. W. L., Dowdeswell, J. A., Gogineni, S., Howat, I., ... \& Steinhage, D. (2013). A new bed elevation dataset for Greenland., Data available at Greenland 5 km DEM, Ice Thickness, and Bedrock Elevation Grids, Version 1, http://nsidc.org/data/nsidc-0092, accessed 4 September 2018.}  
	\label{fig:GISxsection}
\end{figure}

\begin{figure}[p]
	\centering
  \includegraphics[width=7 in]{gis_physical}%
	\caption{The Greenland Ice Sheet seen in map view (Panel A, left) and in a satellite image indicating the speed of ice flow (Panel B, right). In panel B, faster speeds are shown by reds and purples. The three largest outlet glaciers on the island are indicated on both panels, and are clearly shown by the red and purples of rapidly flowing glaciers. Images courtesy of NASA. }  
	\label{fig:gis_physical}
\end{figure}

\begin{figure}[p]
	\centering
  \includegraphics[width=5 in]{SLR}%
	\caption{All three perennial ice reservoirs are melting and contributing to sea level rise. The curves show accumulated contributions to sea level rise since 1990 CE, projections are shown with dotted lines. Mountain glaciers are contributing the most now, roughly \SI{0.8}{\milli\metre} per year, while Greenland and Antarctica contribute a little less. The inset shows the total contribution of all three, which surpasses \SI{54}{\milli\metre} per year by 2020: 2 inches. By 2025 CE or so, melting Cryosphere will be adding more than an inch to sea levels every decade, and that value will increase over time. Shading shows very likely (90\%) confidence intervals for the values. Data digitized from AR 5, Ch 4, Figure 4.25. Extrapolations are OLS quadratic fits to the digitized data.}  
	\label{fig:SLR}
\end{figure}

\begin{figure}[p]
	\centering
  \includegraphics[width=5 in]{SLR_vs_SLE}%
	\caption{The size of the reservoir matters less than the warming of a reservoir, at least as it applies to sea level rise. Above, the blue bars show the proportion of sea level equivalent in each of the three perennial cryosphere reservoirs; Antarctica has the vast majority of the ice, nearly 90\%. Mountain glaciers have only 1\% of \emph{potential} sea level equivalent, but contributes 50\% of observed sea level rise through 2020 CE. The reason is straight-forward: mountain glaciers are warming far faster than the more polar ice sheets. This graph was suggested by Meier, M.F., M.B. Dyurgerov, U.K. Rick, S. O'Neel, W.T. Pfeffer, R.S. Anderson, S.P. Anderson, and A.F. Glazovsky. 2007. Glaciers dominate eustatic sea-level rise in the 21st century. Science 317: 1064-1067, but with data from Table \ref{tab:wti} and from the data and forecasts described in Figure \ref{fig:SLR} caption}.   
	\label{fig:SLR_vs_SLE}
\end{figure}

\begin{figure}[p]
	\centering
  \includegraphics[width=5 in]{LarsenB}%
	\caption{The rapidity and scale of ice shelf loss on the Antarctic Peninsula is well-documented in this series of satellite images of the Larsen B ice shelf. The images start on January 31, 2002 CE (upper left) and end just 6 weeks later on March 17. In the first image, lines of blue melt-water pools indicate the floating ice shelf is in distress, as melt water pools into existing cracks in the \SI{220}{\metre} (720 foot) thick ice shelf. This water rapidly funnels through the ice shelf, wedging the cracks open and (second image) leading to the rapid disintegration of the shelf into thousands of ice bergs. By March 17, the ice shelf had retreated by \SIrange{50}{60}{\kilo\metre} in less than 45 days. The final image shows the ice shelf's limits in 1947 (violet), 1961 (indigo) and 1993 (blue), which indicates that the shelf had been relatively stable prior to 1993 CE. The red, orange, yellow and green lines indicate the shelf edge over the collapse. The dotted line shows the outline of the State of Rhode Island and Providence Plantations, the smallest state (with the longest name) in the United States. Sea level did not rise when the ice shelf broke up; floating ice displaces a volume of water equal to its melted volume. Sources: Outlines for 1947, 1961, and 1993 are courtesy of Ted Scambos, National Snow and Ice Data Center, University of Colorado, Boulder, at https://nsidc.org/news/newsroom/larsen\_B/index.html. Outlines for other times are by the author from the imagery. The imagery is from NASA/Goddard Space Flight Center, https://svs.gsfc.nasa.gov/30160. Image at https://svs.gsfc.nasa.gov/vis/a030000/a030100/a030160/woc\_CollapseLarsen-B\_IceShelf\_Qsign.png}.   
	\label{fig:LarsenB}
\end{figure}

\section{Tables}   \label{Cryosphere_Tables}
.

\begin{sidewaystable} [pht]
\begin{threeparttable}
\label{tab:wti}
\centering
\caption{Where's the ice?}
\begin{tabular}{@{}lrrrrrrr@{}} \toprule
&\multicolumn{5}{c}{Extent}	& 	\multicolumn{2}{c}{Annual Change} \\ \cmidrule(r){2-6} \cmidrule(r){7-8}
Perennial & Area (\%)& Area ($Mkm^{2}$)	& Volume ($Mkm^{3}$)& Thickness (m)	& SLE (m) & Volume ($km^{3}$)& SLE (mm) \\ \midrule
Antarctica	&8.8	&12   &23  	&1892	&58.3	&0.001	&0.6\\
Greenland		&1.2	&1.8	&2.9	&1652	&7.4	&0.01	  &0.5\\
Glaciers  	&0.5	&0.74	&0.16	& 221	&0.41	&0.2	  &0.7\\
Permafrost	&10.5	&18	  &0.02	&1	  &0.1	&-	    &-  \\ 
\emph{Total}&21.0	&33	  &26 	&-	  &66.2	&-	    &1.8\\ \midrule
&\multicolumn{5}{c}{Extent}	&\multicolumn{2}{c}{Annual Change} \\ \cmidrule(r){2-6} \cmidrule(r){7-8}
Seasonal  & Average (\%) & Average ($Mkm^{2}$)  & Min ($Mkm^{2}$) & Max ($Mkm^{2}$)	& Thickness (m) & Extent ($km^{2}$)& Thickness (m) \\ \midrule
$\ \ $Sea Ice   	& & &	&	&	&	&\\
$\ \ \ $Northern	&2.8	&10	&6.2	&1.4E7	&2	&-0.4	&-0.06\\
$\ \ \ $Southern	&3.0	&11	&2.9	&1.9E7	&1	&0.2	&-\\
$\ \ $Snow      	& & &	&	&	&	&\\
$\ \ \ $Northern	&1-30	&11	  &1.9	&4.5E7	&-	&-0.4	&-\\
$\ \ \ $Southern	&10 	&14.3	&13.6	&15	&-	&~0	&-\\ \bottomrule
\end{tabular}
    \begin{tablenotes}
		\small
      \item SLE is sea level equivalent, the global average sea level rise from melting all the ice in each reservoir. The total does \emph{not} include floating ice.
			\item The ``M'' in $Mkm^{2}$ and $Mkm^{3}$ represents 1 million units.
			\item Snow cover in the Southern Hemisphere is largely on glacial and sea ice in Antarctica, and doesn't substantially change Earth's albedo. Ice in the Northern Hemisphere, on the other hand, falls on seasonally ice-free areas.
    \end{tablenotes}

\end{threeparttable}
\end{sidewaystable}


\end{document}
